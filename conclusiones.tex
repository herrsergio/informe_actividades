% Conclusiones
% Sergio Cuellar
% 28 sep 2011

\chapter{Conclusiones}
\label{chap:conclusiones}

La experiencia obtenida al laborar en Premium Restaurant Brands desarrollando sistemas ha sido de gran valor para mi, ya que he desarrollado más mis habilidades de programación, así como también he adquirido conocimiento de la industria de restaurantes que antes desconocía. Los conocimientos obtenidos en la universidad, únicamente sirven de base para el desarrollo que uno tenga como profesionista y es de particular importancia ampliar estos conocimientos y tenerlos siempre actualizados, ya que como sabemos, la tecnología avanza rápidamente y surgen nuevas herramientas, nuevo hardware, nuevos lenguajes de programación, etc.

El trabajo en equipo y de tipo multidisciplinario es fundamental para poder llevar a cabo los desarrollos requeridos, la colaboración implica alcanzar metas comunes y entender los requerimientos, es decir estructurar las necesidades del negocio. La comunicación constante, con las diversas áreas involucradas en el proyecto tendrá como resultado un proyecto exitoso y acorde con las necesidades del cliente.

De los trabajos descritos en este informe de actividades, se puede concluir que éstos han logrado un gran ahorro de tiempo en las actividades de los gerentes en los restaurantes, lo que implica que puedan ocuparse en otras actividades pendientes, directas en la operación del restaurante. El adecuado uso de las tecnologías de la información permite que los datos proporcionados al momento y de manera veraz contribuyan a la toma de decisiones por parte del gerente de restaurante y del área de Operaciones. Por ejemplo, con el reporte de ``Estimado de Usos Ideales'' el gerente puede prever la cantidad de producto que se puede consumir, y estar preparado, solicitando a los proveedores más producto de acuerdo al inventario.

Mediante el uso del lector de huellas digitales, se espera un ahorro en el concepto de pago de nómina, ya que ésta se paga de acuerdo al número de horas trabajadas. Contabilizando de manera exacta y sin problemas de suplantación, las horas pagadas serán realmente las justas. Además de que la integración del sistema punto de venta con el dispositivo se realiza de manera transparente sin el uso de software propietario, lo que implica ahorro de dinero, al evitar el pago de licencias o desarrollos externos.

Una de las características que he visto al laborar en el área de sistemas, es que existen varios proyectos de una o distintas áreas, incluida la misma de sistemas, que conlleva a desarrollar un alto nivel de responsabilidad y organización, para poder terminar los desarrollos en el tiempo establecido. Una de las áreas de oportunidad que tiene Sistemas en Premium Restaurant Brands es la implementación de un modelo de procesos que le ayude al desarrollo de los proyectos de una forma más estructurada. Sí es verdad que todos los proyectos se han llevado a cabo, hasta su finalización, el proceso de desarrollo se puede mejorar. La madurez de un proceso es el nivel al cual está explícitamente documentado, gestionado, medido, controlado y continuamente mejorado. Un modelo de procesos es un conjunto estructurado de elementos que describen las características de procesos efectivos y de calidad, indicando ``qué hacer'', no ``cómo hacer'' ni ``quién hace''. Modelos de procesos como ISO o MoProSoft pueden ayudar a la mejora en el desarrollo y gestión de los proyectos a través de estandarización de prácticas, evaluación e integración de mejoras continuas. Las actividades que se deben desarrollar al implementar un modelo de gestión de procesos, puede resultar en una  burocratización de éstas, lo cual es una desventaja. Por lo que puede convenir tomar ideas de estos modelos para obtener lo mejor de ellos, sin llegar al papeleo que puede acarrear.
