% Resultados
% Sergio Cuellar
% 28 Enero 12
% Reorganización del informe

\chapter{Resultados}
\label{chap:resultados}

%\section{Auditoría Electrónica}
%\label{sec:res_auditoria}

La eliminación del impresor de auditoría trajo consigo muchos beneficios los cuales impactaron económicamente a la empresa. Una oficina \textit{paperless} es aquella en donde el uso de papel se ha eliminado o su uso ha disminuido en gran medida. Con esta iniciativa se eliminó la compra de rollos de papel usados por el impresor, la adquisición y mantenimiento de impresores y cintas para éstos. El número de tickets en un día promedio por restaurante es de alrededor de trescientos, además de que todos éstos se deben imprimir para entregárselo al cliente, el impresor de auditoría también hacía una impresión simplificada de cada ticket.

Por cuestiones de auditoría de Hacienda, todos los rollos de papel con los tickets de auditoría se debían guardar por un periodo determinado de tiempo, lo que ocasionaba gastos de almacenaje y transporte.

Ahora la manera de consultar todos estos tickets es a través de un servidor web centralizado para la consulta por restaurante y por día. El ahorro anual de este proyecto es de alrededor MXN \$700 mil anuales.

%\section{Usos Ideales en KFC}
%\label{sec:res_uso_ideal}

Gracias al reporte de Estimado de Usos Ideales para KFC, la operación en el restaurante se encuentra más preparada para tener el nivel adecuado de inventario de los productos críticos en el restaurante. Así mismo, la merma se ha visto disminuida, ya que se puede saber con anticipación, por ejemplo la cantidad de papas a la francesa, cuántos kilos de col, cuántos paquetes de marinador, etc. se utilizarán. Una de las claves para el correcto funcionamiento de los usos ideales es que las recetas de los productos utilicen la cantidad exacta de ingredientes que se indican. Ya que en caso contrario el uso ideal puede tener cierto margen de error. Gracias a la ayuda que se le proporciona a la operación del restaurante, este reporte lo migré a Pizza Hut por solicitud del equipo operativo de la marca.

La captura de cierre de lote ha ayudado a que el staff gerencial del restaurante esté al pendiente de realizar cada día al finalizar la operación, el cierre de lote en cada terminal bancaria. Esto con la finalidad de que el banco deposite el dinero al siguiente día hábil, agilizando las tareas contables y financieras de la empresa. Con el envío automático de un correo electrónico en caso de que la terminal tenga algún problema al realizar el cierre de lote, los soportes proporcionados por el proveedor se realizan de manera rápida y efectiva, evitando en mayor medida que el gerente tenga que llamar directamente al proveedor para levantar un ticket de servicio. 

%\section{Control de Asistencia}
%\label{sec:res_control_asistencia}

Actualmente el proyecto de control de asistencia mediante lector biométrico se encuentra en pruebas en dos restaurantes, uno de cada marca. El proyecto permite al departamento de Nóminas:

\begin{itemize}
 \item Tener un control sistematizado de la asistencia en restaurantes.
 \item Justificaciones para el otorgamiento de prestaciones.
 \item Control de incapacidades y permisos del personal.
 \item Posibilidad de ampliar la emisión de nuevos reportes de labor.
 \item Asegurar al 100\% la integridad del pago de premio por asistencia.
 \item Control adecuado de ausentismo para descontarse de las cuotas obrero patronales.
\end{itemize}

El beneficio que tiene el área de Operaciones junto con el gerente del restaurante:

\begin{itemize}
 \item Visualización de la asistencia y staff gerencial.
 \item Checado diario de entrada/salida.
 \item Programación de descansos semanales.
 \item Control de retardos, faltas e incapacidades.
 \item Control de horarios truncos.
\end{itemize}

Para Recursos Humanos:

\begin{itemize}
 \item Enrolamiento a partir del ingreso.
 \item Cambios de estructura / movilidad de personal.
 \item Incapacidades y ausentismo.
 \item Excepciones de pago.
\end{itemize}

Uno de los mayores obstáculos que tuvo el proyecto en las primeras semanas de operación fue la renuencia por parte de los asociados de ingresar su entrada y/o salida a través del dispositivo biométrico. Poco a poco ha aumentado la disciplina de realizar esta tarea, ya que en caso de que olviden ingresar su salida, no hay manera de poder contabilizar el tiempo laborado para generar la nómina.

De acuerdo a cálculos preliminares realizados por parte del equipo de Nómina, se tiene contemplado un ahorro de MXN \$3 millones anuales en pago de cuotas obrero patronales en primera instancia, más el ahorro en la nómina gracias a tiempos reales de labor.

