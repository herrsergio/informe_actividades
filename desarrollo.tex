% Desarrollo
% Sergio Cuellar
% 30 de abril 2011

\chapter{Desarrollo}
\label{chap:desarrollo}

El área de Desarrollo de Sistemas de Restaurantes está conformado por cuatro personas encargadas de proveer varios servicios de apoyo a alrededor de 340 restaurantes propios de la franquicia de PRB, así como alrededor de 150 restaurantes de otros franquiciatarios. Los requerimientos de desarrollos nuevos son proporcionados generalmente por el área de Operaciones cuando éstos tienen que ver la operación misma del restaurante, o bien, por la misma área cuando se deben realizar actualizaciones, corrección de bugs, etc., al sistema.

\section{Objetivos}
\label{sec:objetivos}

Entre los servicios proporcionados por el área de Desarrollo de Sistemas de Restaurantes se encuentran:

\begin{itemize}
 \item Desarrollo y mantenimiento de reportes solicitados por el área de Operaciones.
 \item Mantenimiento al punto de venta (programación de mejoras, nuevas funcionalidades, corrección de bugs).
 \item Programación de herramientas que coadyuven a la eficaz operación del restaurante.
 \item Desarrollo de aplicaciones que contribuyan a disminuir el costo de operación.
 \item Investigación de nueva infraestructura de sistemas en el restaurante (impresoras, tarjetas de vídeo, terminales, etc.)
\end{itemize}

\section{Descripción del Sistema en restaurantes}
\label{sec:descripcion}

Cada uno de los restaurantes cuenta con una computadora, en la cual se efectúan diversos procesos como son:

\begin{itemize}
 \item Registro y procesamiento de la venta.
 \item Generación de reportes.
 \item Manejo de periféricos como impresora de tickets, impresora del gerente, cajas registradoras, terminales, etc.
 \item Actividades gerenciales (correo electrónico, consultas web de la intranet, uso de procesador de textos y hoja de cálculo, etc).
\end{itemize}

\subsection{Software utilizado}
\label{subsec:software}

Una de las características principales del sistema en restaurantes, es que casi en su totalidad, a excepción del punto de venta que fue desarrollado por la compañía, es software libre.

\subsubsection{Software Libre}
\label{sec:software_libre}

El software libre es software que puede ser usado, estudiado y modificado sin restricción y el cual puede ser copiado y redistribuido en su forma modificada u original, ya sea sin restricciones o con mínimas restricciones que aseguren que los siguientes usuarios del software puedan seguir haciendo estas actividades. El software libre es generalmente disponible sin cargo alguno pero puede tener algunas cuotas, por ejemplo para ser distribuido en forma de CDs u otras formas.

En la práctica, para que un software sea distribuido como software libre, el código fuente de éste debe estar disponible para el usuario, así como también indicarle en una nota que se le conceden los derechos arriba mencionados. Esta nota puede ser, ya sea, una licencia de software libre o indicando que el código fuente está disponible al dominio público.

Las ventajas del software libre son:

\begin{itemize}
 \item Bajo costo, lo que implica el ahorro en pago de licencias.
 \item Es posible adaptar el software a las necesidades que tenga cada usuario, teniendo como resultado un software personalizado.
 \item Al ser público el código fuente, permite que programadores hagan correcciones a errores y mejoren el software de manera rápida.
 \item De acuerdo a los dos últimos puntos, el software libre no depende de una única empresa u organización, lo que evita que se impongan condiciones en su uso, como ocurre con el software privado.
 \item La innovación tecnológica que surge gracias a que cada usuario puede hacer aportaciones a la mejora del software.
\end{itemize}

El sistema en los restaurantes se compone de:

\begin{itemize}
 \item Sistema Operativo GNU/Linux kernel 2.6, basado en Knoppix con escritorio Xfce.
 \item Apache Tomcat
 \item Servidor HTTP Apache
 \item PostgreSQL
 \item Mozilla Firefox
 \item Mozilla Thunderbird
 \item OpenOffice.org
 \item Herramientas GNU (bash, ksh, awk, perl, etc.)
 \item Sistema punto de venta SUS/FMS (propietario, desarrollado por la misma compañía.
\end{itemize}
