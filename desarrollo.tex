% Desarrollo
% Sergio Cuellar
% 30 de abril 2011

\chapter{Desarrollo}
\label{chap:desarrollo}

El área de Desarrollo de Sistemas de Restaurantes está conformado por cuatro personas encargadas de proveer varios servicios de apoyo a alrededor de 340 restaurantes propios de la franquicia de PRB, así como alrededor de 150 restaurantes de otros franquiciatarios. Los requerimientos de desarrollos nuevos son proporcionados generalmente por el área de Operaciones cuando éstos tienen que ver la operación misma del restaurante, o bien, por la misma área cuando se deben realizar actualizaciones, corrección de bugs, etc., al sistema.

\section{Objetivos}
\label{sec:objetivos}

Entre los servicios proporcionados por el área de Desarrollo de Sistemas de Restaurantes se encuentran:

\begin{itemize}
 \item Desarrollo y mantenimiento de reportes solicitados por el área de Operaciones.
 \item Mantenimiento al punto de venta (programación de mejoras, nuevas funcionalidades, corrección de bugs).
 \item Programación de herramientas que coadyuven a la eficaz operación del restaurante.
 \item Desarrollo de aplicaciones que contribuyan a disminuir el costo de operación.
 \item Investigación de nueva infraestructura de sistemas en el restaurante (impresoras, tarjetas de vídeo, terminales, etc.)
\end{itemize}

\section{Descripción del Sistema en restaurantes}
\label{sec:descripcion}

Cada uno de los restaurantes cuenta con una computadora, en la cual se efectúan diversos procesos como son:

\begin{itemize}
 \item Registro y procesamiento de la venta.
 \item Generación de reportes.
 \item Manejo de periféricos como impresora de tickets, impresora del gerente, cajas registradoras, terminales, etc.
 \item Actividades gerenciales (correo electrónico, consultas web de la intranet, uso de procesador de textos y hoja de cálculo, etc).
\end{itemize}

\subsection{Software utilizado}
\label{subsec:software}

Una de las características principales del sistema en restaurantes, es que casi en su totalidad, a excepción del punto de venta que fue desarrollado por la compañía, es software libre.
