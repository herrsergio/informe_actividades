% Introduccion
% Sergio Cuellar
% 26 Marzo 2011

\chapter{Introducción}

\label{chap:intro}

La tecnología de la información (TI) es una herramienta valiosa dentro de la estrategia de negocios de Premium Restaurants Brand (PRB). En el departamento de Sistemas proveemos de soluciones tecnológicas a nuestros restaurantes, así como también establecemos estándares tecnológicos y de procesos en nuestras oficinas.

PRB maneja las marcas KFC y Pizza Hut, contando con más de 300 restaurantes a nivel nacional. El sistema operativo que se maneja en los restaurantes es GNU/Linux, teniendo como punto de venta (POS\footnote{Point of Sale}), un desarrollo propietario realizado en su origen por Yum Restaurant Brands de Estados Unidos.

Las Tecnologías de la Información apropiadas, cuando se utilizan junto con los principios de administración de ingresos, puede ayudar a los restaurantes a aumentar éstos y sus ganancias. Las Tecnologías de la Información han revolucionado el panorama de los negocios en el mundo y la industria de los restaurantes no es la excepción. Las TI han modificado a varias industrias y ahora juegan un papel fundamental en las reglas que rigen el mundo de los negocios y en la forma en que se acercan a los clientes. Las ventajas de las TI en cuanto al incremento de la competitividad, reducción de errores y creación de nuevas funcionalidades son incuestionables en cualquier sector, incluido el de los restaurantes.

Los restaurantes han invertido en automatización de funciones relacionadas con la nómina, contabilidad, inventario, etc., con el objetivo de bajar costos. También existen los sistemas punto de venta (POS, Point of Sale), los cuales facilitan sustancialmente la operación del restaurante, ayudando a realizar el marcaje de productos de manera rápida, con lo que se logra atender un mayor número de clientes en menor tiempo, la modificación automatizada de los menús, cambio de precios, etc.

En PRB contamos con un POS creado a la medida, lo cual permite a la empresa realizar desarrollos que implican la modificación del mismo POS, ya que se cuenta con el código fuente de éste. Estas modificaciones implican programación de nuevas funcionalidades o de corrección de errores (bugs). Así mismo es posible realizar desarrollos que se relacionen con el POS e interactúen con él, para la obtención y manipulación de datos generados por el punto de venta.

Diversas áreas de la empresa, como Operaciones y Finanzas, principalmente, considera los datos de la venta como muy importantes, ya que gracias a ellos pueden saber la tendencia que siguen las ventas, el comportamiento de los nuevos productos y/o promociones, los niveles de inventarios, los días y horas de mayores ventas, etc. 

Una de las actividades del área de Desarrollo de la Dirección de Sistemas, se encarga de la programación de reportes web de los datos que ayuden principalmente a los gerentes de restaurante a visualizar los datos de las operaciones de su restaurante de manera casi inmediata.

\section{Objetivos}
\label{sec:objetivosss}

Desarrollo e implementación de reportes que ayuden a la operación de restaurantes a eficientar sus procesos, conocer resultados, ahorrar tiempo, identificar áreas de oportunidad. Así como también el desarrollo de aplicaciones que ayuden a la mejora en el sistema en general del punto de venta utilizado.

La plataforma a utilizar es GNU/Linux utilizando Apache Tomcat, Java, JSP, Perl, C, Bash, y otras herramientas de software libre.