\begin{appendices}

\chapter{Cambios a \texttt{libfprint}}

Realicé los siguientes cambios al archivo \texttt{data.c} de la librería \texttt{libfprint} para que pueda registrar y reconocer a más de un usuario. A continuación se muestra la salida del comando:

\begin{Verbatim}
 diff -u data.c.orig data.c
\end{Verbatim}

El cual sirve para poder distribuir únicamente el cambio sin necesidad de mandar el archivo completo. A esto se le llama un parche, el cual se aplica de la siguiente manera:

\begin{Verbatim}
 patch < data.patch
\end{Verbatim}


\begin{Verbatim}[fontsize=\small]
--- data.c.orig	2011-05-10 16:36:10.000000000 +0000
+++ data.c	2011-05-10 16:36:10.000000000 +0000
@@ -49,8 +49,9 @@
 
 static char *base_store = NULL;
 
-static void storage_setup(void)
+static void storage_setup(char *path, char *user)
 {
+    /*
 	const char *homedir;
 
 	homedir = g_getenv("HOME");
@@ -58,8 +59,9 @@
 		homedir = g_get_home_dir();
 	if (!homedir)
 		return;
+    */
 
-	base_store = g_build_filename(homedir, ".fprint/prints", NULL);
+	base_store = g_build_filename(path, user, "fprint/prints", NULL);
 	g_mkdir_with_parents(base_store, DIR_PERMS);
 	/* FIXME handle failure */
 }
@@ -225,7 +227,7 @@
  * \returns 0 on success, non-zero on error.
  */
 API_EXPORTED int fp_print_data_save(struct fp_print_data *data,
-	enum fp_finger finger)
+	enum fp_finger finger, char *pathd, char *user)
 {
 	GError *err = NULL;
 	char *path;
@@ -235,7 +237,9 @@
 	int r;
 
 	if (!base_store)
-		storage_setup();
+		storage_setup(pathd, user);
+
+    //printf("fp_print_data_save: %s\n", base_store);
 
 	fp_dbg("save %s print from driver %04x", finger_num_to_str(finger),
 		data->driver_id);
@@ -334,14 +338,17 @@
  * \returns 0 on success, non-zero on error
  */
 API_EXPORTED int fp_print_data_load(struct fp_dev *dev,
-	enum fp_finger finger, struct fp_print_data **data)
+	enum fp_finger finger, struct fp_print_data **data,
+    char *pathd, char *user)
 {
 	gchar *path;
 	struct fp_print_data *fdata;
 	int r;
 
-	if (!base_store)
-		storage_setup();
+	//if (!base_store)
+		storage_setup(pathd, user);
+
+    //printf("fp_print_data_load: %s\n", base_store);
 
 	path = get_path_to_print(dev, finger);
 	r = load_from_file(path, &fdata);
@@ -551,7 +558,7 @@
  * \returns a NULL-terminated list of discovered prints, must be freed with
  * fp_dscv_prints_free() after use.
  */
-API_EXPORTED struct fp_dscv_print **fp_discover_prints(void)
+API_EXPORTED struct fp_dscv_print **fp_discover_prints(char *pathd, char *user)
 {
 	GDir *dir;
 	const gchar *ent;
@@ -563,7 +570,9 @@
 	unsigned int i;
 
 	if (!base_store)
-		storage_setup();
+		storage_setup(pathd, user);
+
+    //printf("fp_discover_prints: %s\n", base_store);
 
 	dir = g_dir_open(base_store, 0, &err);
 	if (!dir) {
\end{Verbatim}

\chapter{Modificación a tabla \texttt{pp\_emp\_check}}

Ya existía la tabla \texttt{pp\_emp\_check}, pero le realicé algunas modificaciones para soportar horas truncas, identificar si fue registro manual o mediante el lector biométrico.

\begin{Verbatim}[fontsize=\tiny]
--
-- PostgreSQL database dump
--

SET statement_timeout = 0;
SET client_encoding = 'LATIN1';
SET standard_conforming_strings = off;
SET check_function_bodies = false;
SET client_min_messages = warning;
SET escape_string_warning = off;

SET search_path = public, pg_catalog;

ALTER TABLE ONLY public.pp_emp_check DROP CONSTRAINT pp_emp_check_pkey;
DROP TABLE public.pp_emp_check;
SET search_path = public, pg_catalog;

SET default_tablespace = '';

SET default_with_oids = false;

--
-- Name: pp_emp_check; Type: TABLE; Schema: public; Owner: postgres; Tablespace: 
--

CREATE TABLE pp_emp_check (
    store_id smallint DEFAULT 0 NOT NULL,
    date_id date NOT NULL,
    emp_num character varying(6) NOT NULL,
    timein1 timestamp without time zone,
    timeout1 timestamp without time zone,
    timein2 timestamp without time zone,
    timeout2 timestamp without time zone,
    entry_type character varying(5)
);


ALTER TABLE public.pp_emp_check OWNER TO postgres;

--
-- Name: COLUMN pp_emp_check.entry_type; Type: COMMENT; Schema: public; Owner: postgres
--

COMMENT ON COLUMN pp_emp_check.entry_type IS 'Ingreso manual o con lector de huellas';


--
-- Name: pp_emp_check_pkey; Type: CONSTRAINT; Schema: public; Owner: postgres; Tablespace: 
--

ALTER TABLE ONLY pp_emp_check
    ADD CONSTRAINT pp_emp_check_pkey PRIMARY KEY (store_id, date_id, emp_num);


--
-- PostgreSQL database dump complete
--

\end{Verbatim}

\chapter{Procedimiento almacenado para cálculo de horas}

Este procedimiento almacenado o \textit{store procedure} lo realicé para el cálculo de las horas totales almacenadas desplegadas en el reporte de ``Reporte de Asistencias en biométrico'' en eReports.

\begin{Verbatim}[fontsize=\tiny]
CREATE OR REPLACE FUNCTION total_worked_hours(character varying, character varying, character varying, character varying)
  RETURNS character varying AS
$BODY$
DECLARE
    timein1C   ALIAS FOR $1;
    timeout1C  ALIAS FOR $2;
    timein2C   ALIAS FOR $3;
    timeout2C  ALIAS FOR $4;
    timein1  timestamp without time zone;
    timein2  timestamp without time zone;
    timeout2 timestamp without time zone;
    timeout1 timestamp without time zone;
    result1 interval;
    result1C varchar(4);
    result2 interval;
    result2C varchar(4);
    result  interval;
    finalresult varchar(20);
BEGIN

    IF timein1C != '' THEN
        IF timeout1C != '' THEN
            timein1  := CAST(timein1C  as timestamp);
            timeout1 := CAST(timeout1C as timestamp);
            result1  := timeout1 - timein1;
        ELSE
            result1C := '0';
        END IF;
    ELSE
        result1C := '0';
    END IF;

    IF timein2C != '' THEN
        IF timeout2C != '' THEN
            timein2  := CAST(timein2C  as timestamp);
            timeout2 := CAST(timeout2C as timestamp);
            result2  := timeout2 - timein2;
        ELSE
            result2C := '0';
        END IF;
    ELSE
        result2C := '0';
    END IF;

    IF result1C = '0' THEN
        IF result2C = '0' THEN
            finalresult := '';
        ELSE 
            result := result2;
            finalresult := CAST(result as varchar);
        END IF;
    ELSE
        IF result2C = '0' THEN
            result := result1;
            finalresult := CAST(result as varchar);
        ELSE
            result := result1 + result2;
            finalresult := CAST(result as varchar);
        END IF;
    END IF;
  
    
    return finalresult;
   
END;$BODY$
  LANGUAGE plpgsql VOLATILE
  COST 100;
ALTER FUNCTION total_worked_hours(character varying, character varying, character varying, character varying)
  OWNER TO postgres;
\end{Verbatim}


\end{appendices}